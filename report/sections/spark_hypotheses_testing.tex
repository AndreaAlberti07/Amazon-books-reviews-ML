\section{Spark Hypotheses Testing}
To showcase the feasibility of implementing data analysis within a Big Data context, we opted to replicate some hypothesis testing using 
Spark, focusing particularly on Hypotheses 1 and 3.

\subsection*{Hypothesis 1}
Addressing this hypothesis involved several key steps:

\begin{itemize}[leftmargin=*, noitemsep]
    \item \textbf{Compute the Helpfulness Score:}
    This was straightforwardly achieved by leveraging the `WithColumn` method of the Spark DataFrame, creating a new column with the updated values.
    
    \item \textbf{Compute the Text Length:}
    Text length computation was accomplished by utilizing `regexp\_replace` to eliminate punctuation, along with `Tokenizer` and `StopWordsRemover` 
    to tokenize the text and remove stop words. Subsequently, a new column containing the text length was generated.
    
    \item \textbf{Bucketize the Text Length:}
    To address this requirement, we utilized the `Bucketizer` class from Spark MLlib in conjunction with a User Defined Function (UDF) to 
    assign appropriate labels to the classes.
    
    \item \textbf{Compute the Correlation Coefficient:}
    Finally, the correlation coefficient was computed using the `Correlation.corr` method from Spark MLlib, specifically the Spearman correlation 
    coefficient. The data was reshaped to conform to the required format using `VectorAssembler`.
    
\end{itemize}

\subsection*{Hypothesis 3}
This hypothesis involved computing the helpfulness score and correlation coefficient, both of which were calculated using the same methods 
described in the previous hypothesis.

\subsection*{Results}
In both test cases, the results closely mirrored those obtained in the local environment.

